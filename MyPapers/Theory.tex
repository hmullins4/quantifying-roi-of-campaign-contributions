\section*{3 Theoretical Framework}

\indent Several complementary perspectives frame how spending translates into electoral outcomes. First, microeconomic models interpret campaign spending as an input into a ``political production function'' or a ``political factory,'' where money buys votes \cite{jacobson1983}. Game-theoretic extensions cast these expenditures as bids in an ``all-pay auction,'' predicting sunk spending and strategic bidding \cite{Levin2004}. These approaches together forecast diminishing returns once campaigns saturate available media markets and core constituencies, and anticipate strategic spending patterns.

\indent Public choice theory refines the microeconomic view by emphasizing the incentives driving both donors and politicians. Contributions can be interpreted as investments --- donors channel resources into campaigns expecting future policy payoffs --- or as rent-seeking bids --- expenditures are sunk costs with no guaranteed return unless the supported candidate wins \cite{becker1983}. Under this lens, contributions reflect a rational actor's calculation of expected policy benefits net of bid expenditures; firms and interest groups will only invest in campaigns when anticipated returns exceed the cost of contributions. 

\indent Next, political science theories focus on how spending shapes voter behavior directly. Spatial utility models hold that voters choose the candidate closest to their own ideal point on an ideological spectrum \cite{downs2024}. Campaign resources, then, enhance a candidate's ability to convey issue positions accurately --- shifting perceived ideological distances and thus vote share. Similarly, persuasion theory argues that expenditures fund targeted messaging that can reframe issues, prime particular considerations, and activate latent preferences \cite{lupia2021}. Both perspectives suggest that money's effect is mediated through communication quality, media mix, and timing. 

\indent Finally, institutional theory adds another dimension by demonstrating how legal and regulatory structures alter the incentives and strategies surrounding campaign finance. Regulatory impact theory posits that contribution limits, disclosure requirements, and public-funding options change the ``production technology'' of campaigns by restricting inputs, increasing the cost of compliance, or shifting resources toward less-regulated vehicles \cite{stratmann2005}. For example, stringent contribution caps may drive donors toward independent expenditures or dark-money networks, reshaping the competitive landscape even if overall spending levels remain unchanged.

\indent Together, these theoretical lenses suggest a few key expectations for empirical analysis: diminishing returns, context-dependence, and strategic allocation. As established in the previous section, marginal vote gains per dollar spent, past a certain point in time, decline at higher spending levels. Moreover, the effectiveness of money varies by incumbency status, district competitiveness, and regulatory regime, also confirmed by earlier studies. Lastly, donors and campaigns allocate resources to maximize expected policy or electoral payoff, leading to heterogeneous spending patterns across races and over time. In the empirical work that follows, I draw on these frameworks to guide model specification and to interpret my estimates of the ROI of an additional million dollars in U.S.\ House political campaigns.
