\section*{1 Introduction}

\indent Over the past fifty years, a succession of landmark Supreme Court decisions has reshaped the role of money in U.S.\ elections. In \textit{Buckley v.\ Valeo} (1976), the Court held that contribution limits --- though a restriction on political speech --- were justified by the government's interest in preventing corruption and preserving electoral integrity, yet it struck down aggregate expenditure limits because they violate the First Amendment's free-speech protections \cite{buckley1976}. More than three decades later, \textit{Citizens United v.\ FEC} (2010) overturned \textit{Austin v.\ Michigan State Chamber of Commerce} (1990) and parts of \textit{McConnell v.\ FEC} (2003), ruling that corporations and unions have a protected right to make independent expenditures in elections, a decision that ushered in the era of Super PACs and ``dark money'' (money given by nonprofits that are not mandated to disclose their donors) \cite{citizensunited2010}. In \textit{McCutcheon v.\ FEC} (2014), the Court further invalidated biennial aggregate limits on individual contributions (at the time, capped at \$123,200 per cycle\footnote{In 2014 USD.}), thereby granting wealthy donors greater flexibility in allocating funds across candidates and committees \cite{mccutcheon2014}. 

\indent These rulings, along with state-level reforms and evolving disclosure regimes, have produced a landscape in which electioneering mobilizes billions in fundraising each cycle. Yet, despite the scale of this spending, political actors, consultancies, and interest groups lack reliable ``return on investment'' (ROI) estimates that convert dollars into votes. From a campaign strategist's perspective, quantifying the marginal effect of an additional \$1 million on win probability or vote share is essential for allocating scarce resources effectively. Likewise, corporate and nonprofit donors need rigorous benchmarks to assess both whether and how to invest in candidates whose policy priorities align with their interests. Finally, voters themselves can benefit from transparent, data-driven insight regarding how spending sways electoral outcomes.

\indent Accordingly, I attempt to answer the following question: \begin{quote} \emph{How do the magnitude and composition of campaign contributions affect a U.S.\ House candidate's probability of winning and expected vote share?} \end{quote} In the following section, I review related empirical literature; in Section 3, I describe the theoretical frameworks, while in Section 4, I describe the database and data-processing steps. In Section 5, I present the econometric and machine-learning specifications, and in Section 6, I report results, interpreting the findings in Section 7 while discussing limitations in Section 8 and concluding in Section 9. In Appendix A, I document the filtering, cleaning, and inflation-adjustment procedures in detail, while in Appendix B I define variables, and in Appendix C I collect supplemental figures and tables.
