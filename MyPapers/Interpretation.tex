\section*{7 Interpretation}

Although there is a vast amount to unpack from the results of the previous section, the models generated tell a consistent story: late-cycle fundraising, particularly from individual donors, is what really moves the needle on both probability of winning and vote-share in U.S.\ House elections. 

\subsection*{7.1 Logistic Regression}

Area under the curve rises from 0.898 at 360 days before the election to a peak of 0.942 by the day before the election. Across all specified cutoffs, accuracy is stable, ranging from 86.5 percent to 89.9 percent. Recall for class 1 (winners) ranges from 84.5 percent to 91.7 percent, while recall for class 0 (losers) ranges from 80.1 percent to 94.3 percent. Although early-cycle fundraising offers strong predictive power, substantially more signal arrives within the final two months of the campaign (see Figure~\ref{fig:roi-year}). These results are consistent with Jacobson (1978), who showed that late contributions are far more informative about eventual electoral outcomes than early money.

\indent At 360 days, the aggregate ROI odds-ratio equals 0.902, suggesting a negative association between dollars raised that early and winning --- potentially reflecting that underfunded candidates (generally challengers) raise money early but still lose; early-cycle fundraising may be more diagnostic than causal, revealing candidate quality and perceived competitiveness. In fact, Adam Bonica examined the relationship between early fundraising and electoral success, stating that ``focusing on general election contests understates the true effect of fundraising on election outcomes'' \cite{bonica2016}. Bonica found that early fundraising strongly predicted who would win primary races, but did not necessarily predict general election success.

\indent Alternatively, early fundraising may reflect not only candidate quality but also strategic front-loading by well-known incumbents to finance long-lead advertising. In other words, well-funded incumbents may raise early to buy TV in advance, so their ROI looks attenuated simply because they are shifting ``spending time'' rather than ``spending amount.'' When observing the proportion of winners versus losers across time (as shown in Section 8), the number of winners who receive donations/begin spending about a year before Election Day greatly exceeds the number of losers who do the same.

\indent According to Figure~\ref{fig:roi-year}, once one reaches 200 days before the election, ROI becomes positive up through Election Day. This is supported by the individual cutoff results, showing that, while there is variation regarding how much a \$1 million increase in receipts multiplies the odds of winning by, there are benefits to receiving donations late in the game. Stratmann (2005) argued that campaign spending has diminishing marginal returns, but those returns are highly time-sensitive, with late spending far more valuable, as validated by my results.

\indent Observing individual contributions, the ROI odds-ratio grows steadily from 1.07 at 360 days to 1.76 at seven days before leveling off at 1.69 at one day. Alternatively, the ROI of non-connected committee contributions is generally negative or near zero throughout, suggesting that this money is less effective, especially one to two weeks before the election. Another way to interpret this is that non-corporate committees tend to finance targeted voter-contact operations, which have slower lead times, and the payoff may not materialize until very late. Finally, corporate committee ROI is strongly negative at 360 days, but turns positive by 120 days, and remains high as Election Day gets closer. Therefore, it seems connected-committee spending is important for candidates, but not too early on. Individual donations are often a reflection of voter enthusiasm, whereas corporate PAC money may buy less targeted ads and thus have a weaker return on investment. 

\subsection*{7.2 GAM Logistic Models}

Compared to the simple logistic models at each cutoff, GAM-based models yielded a higher AUC overall, indicating non-linearities improve classification. ROI patterns are qualitatively similar, but quantitatively different; although the previous models showed a consistent rise in ROI estimates moving closer to the election day, the GAM model captures non-monotonic effects at later cutoffs (60 and 14 days), reflecting that at certain points diminishing returns exist. As mentioned, this is supported by a variety of research.

When observing Figure~\ref{fig:gam-or}, the dips and spikes in ROI become much more apparent. There are a few explanations as to this volatility. In a presidential-election year, the general election falls on the Tuesday after the First Monday in November. Sixty days before that, around early September, two notable things happen. First, the 60-day ``Electioneering Communications'' window opens: under the Bipartisan Campaign Reform Act (BCRA), any broadcast, cable, or satellite ad that (a) refers to a clearly identified federal candidate and (b) is publicly distributed within 60 days of a general election is defined as an ``electioneering communication'' \cite{fec_communications}. From that point forward, corporations and labor unions are prohibited from funding those ads with their general-treasury funds, and any such ad must be reported to the FEC (with details on who paid for it and how much it cost) \cite{congress_plaw}. Around the same time, both the House and Senate typically conclude their August session and return home to campaign and meet constituents. This ``recess'' gives incumbents an official break from Capitol Hill to focus on outreach back in their districts. It also often markets the start of high-profile campaign events that lead into the final two months of the cycle \cite{schedule}. So, about two months before Election Day, one sees both the formal onset of the FEC's electioneering communications restrictions/reporting requirements and Congress's late-summer recess, which ushers in the final sprint of district-and-state campaigning.

This provides a clear explanation for the odds-ratio estimate produced by the logistic GAM 60 days before the election. While two weeks before the election yields a smaller odds-ratio (1.03) compared to two weeks prior and a week after, it is still positive; however, at 60 days, there is a negative return on investment (with an odds-ratio of 0.9). This differs somewhat from the non-GAM logistic model evaluated previously, where the odds-ratio stayed relatively high within the last two months of the election, even if there was some variation.

One could also hypothesis tactical campaign pauses (major debates, party conventions, and so forth) that temporarily crowd out small-donor appeals, essentially creating ``valleys'' in effectiveness.

\subsection*{7.3 Linear Regression}

$R^2$ never rises past 61 percent, indicating that factors besides money, incumbency, and party-district fixed effects explain the variation in vote share. Total ROI, measured as percentage points (pp) of vote-share per \$1 million, moves from strongly negative at -8.75 pp at 360 days to a peak of 0.89 pp at 30 days, tapering off at 0.62 pp at one day. Early-cycle dollars are often picked up by candidates already weak in the polls, hence the negative coefficent. As the season progresses, fundraising increasingly captures true electoral momentum (corroborated by the binary results).

The regularization models (Ridge and LASSO) give nearly, if not exactly, identical $R^2$ but trim down erratic coefficients. These similar results suggest there is little overfitting in this training; the model's level of complexity is not significantly inflating its performance on the training set. Also, feature importances consistently rank independent expenditures against the candidate and independent expenditures in support of the candidate as top predictors --- underscoring the outsized impact of outside spending. Incumbency, party, district partisanship, average transaction type frequency, and the amount of money (in millions) given by individuals remain extremely strong predictors at each cutoff, as indicated by the feature significance across cutoffs.

\subsection*{7.4 Gam Linear Models}

Cross-validated $R^2$ is lower early but peaks at around 69 percent one day before the election, while hold-out $R^2$ peaks in the two-week window. The change in vote share per \$1 million flips from negative to positive between 240 and 120 days before the election, then becomes negative toward the very end of the election cycle, suggesting over-saturation or that last-minute money may be reactive to impending loss rather than causally effective. Once flexibly accounting for splines in all features, the very last-minute money may not further boost vote share on the linear scale --- consistent with diminishing marginal returns. 

\indent Four things stand out throughout this study: first, across every model, the predictive power and causal return on investment of campaign contributions is time-varying, with negligible or negative returns early. There is more variation as Election Day approaches, with the GAMs indicating more money does not increase one's probability of winning or vote share too close to that date. Next, individual donations steadily gain value as they signal voter enthusiam; non-corporate PAC and other committee money is at best neutral; corporate funds have a lagged but positive effect mid- to late-cycle. Third, even after adjusting for money, incumbency and district partisanship dominate. This echoes Jacobson (2004) who argued that the money-win link is conditional on the underlying political environment. Finally, GAMs reveal thresholds and saturation points that linear models do not pick up, suggesting strategic timing of fundraising appeals can yield disproportionately higher gains. Ultimately, the dynamic timing of ROI is a critical nuance that extends the classic findings of Jacobson and more recent non-linear analyses in the campaign-finance literature.
