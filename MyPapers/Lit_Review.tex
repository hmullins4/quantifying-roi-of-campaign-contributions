\section*{2 Literature}

\indent For decades, many researchers have explored how campaign contributions affect electoral results, with a variety of focuses. In his 1978 study, ``The Effects of Campaign Spending in Congressional Elections,'' Gary Jacobson analyzed the relationship between campaign spending and electoral success, in particular highlighting the marginal effect of spending. Using regression analysis, Jacobson attempted to isolate the influence of campaign expenditures on vote share, finding a nonlinear, positive relationship between campaign spending and electoral success; the marginal effect of spending on vote share diminishes as the amount spent increases. Jacobson's findings challenged the idea that money alone can guarantee electoral victory under the argument that, while spending is influential, other factors such as candidate quality and party affiliation play significant roles in electoral outcomes \cite{jacobson1978}.

\indent To address endogeneity concerns that often arise when studying campaign spending, in his 1994 paper, ``Using Repeat Challengers to Estimate the Effect of Campaign Spending on Election Outcomes in the U.S.\ House," Steven Levitt focused on isolating causal effects. Typically, higher spending correlates with unobservable factors, making it difficult to determine the true effect of spending. Levitt compared ``repeat'' challengers --- candidates who run in consecutive elections (facing the same opponent) --- effectively controlling for variables like candidate quality and district characteristics that remain constant over time. The results show that money does indeed have an effect on electoral outcomes, but the impact is not large; the effectiveness of campaign spending depends significantly on context, such as the challenger's initial electoral position and the incumbent's strength, underscoring the limitations of financing \cite{levitt1994}.

\indent In ``Buying Supermajorities'' (1996), Groseclose and Snyder concentrated on the probability of achieving landslide margins, focusing not just on likelihood of winning but rather winning substantially. From their research, Groseclose and Snyder made many conclusions: campaign spending increases the likelihood of a landslide victory, but the relationship weakens after a certain level of spending; the impact of money varies depending on the strength of the incumbent and the competitiveness of the district; and the effectiveness of spending to secure supermajorities depends on the political context \cite{groseclose1996}.

\indent Building on his marginal-returns analyses, Jacobson (2004) revisited his research with two decades of additional data, introducing district fixed-effects, lagged vote-share controls, and spline specifications to confirm diminishing returns and differential effects for incumbents versus challengers \cite{jacobson2004}: money is a necessary but not sufficient condition --- candidate quality, district lean, and national tides remain dominant in determining vote share gains.

\indent Collectively, these studies share a number of core features, both substantive and meth\-o\-do\-log\-i\-cal. The researchers all employ regression techniques to estimate a ``dollars-to-votes'' relationship, taking explicit steps to control for confounders --- incumbency, district partisanship, candidate quality, and so forth. The research above, amongst others, documents clearly that challengers generally benefit more per dollar and that marginal payoffs shrink at high spending levels. These previous projects provide guidance for my own research.
