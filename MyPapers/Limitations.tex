\section*{8 Limitations and Interpretation Caveats}

To start, because GAMs are nonlinear, the derivative changes with the input values. Consequently, by computing finite differences at the mean, I capture only the average partial effect there --- an estimate that may fail to represent slopes at other points in the covariate domain. Also, all ROI $\beta$s, ORs, and elasticities reported here reflect associational effects --- ultimately, how an additional \$1 million is correlated with higher win odds (or vote share) in our sample, holding other covariates at their means. 

\indent A clear class imbalance exists when predicting winners versus losers across some cutoffs; for example, at $360$ days before the election, proportionally, about 80 percent of the data consists of winners. Intuitively, this makes sense: as Election Day approaches, there is more of a 50:50 split between winners and losers, but earlier on, not everyone has received contributions. The AUC is however still high for each model across each chosen cutoff. Similarly, the recalls for class 1 (winners) and class 0 (losers) are relatively high, with the lowest recall value being for class 0 at 360 days of 0.724. Although the GAMs performed better in terms of AUC, accuracy, and class 1 recall, they performed worse in terms of class 0 recall, indicating that predictive accuracy of the minority class was worse under the generalized additive model. The difference is not drastic, but it is something to note: this is simply a result of how the fundraising process goes.

\indent When observing the odds-ratio trend across a 365 day period for the logistic model, a more-or-less upward trend is apparent. Furthermore, Figure~\ref{fig:gam-or} depicts how the logistic GAM's ROI fluctuate over time. Note, however, the numbers generated for each ROI $\beta$ (across \textit{all} models) in the study are based on arbitrary cutoffs for snapshot comparisons. Although overall trends are implied (and validated by existing literature), there may be greater variation across time than what is seen solely from the eight cutoff days examined. Different or more granular cutoffs might reveal subtler temporal patterns.

\indent A drastic dip occurred in the hold-out $R^2$ at 120 days for the linear GAM, signaling perhaps potential under- or over-smoothing for that slice of data, meaning additional tuning may be necessary to achieve improved results.

\indent $R^2$, for this project, only serves to tell us the percentage of variation in vote share that may be explained by the features within the models. The lower $R^2$s obtained within this project do not automatically disqualify these models as bad or inaccurate nor the estimates as useless. Even with low $R^2$s, one may still draw important conclusions about the relationships between variables, particularly if the independent variables are statistically significant (see Figure~\ref{fig:heatmap}).
